%%%%%%%%%%%%%%%%%%%%%%%%%%%%%%%%%%%%%%%%%%%%%%%%%%%%%%%%%%%%%%%%%%%%%%%%%%%%%%%%
% Author : [Name] [Surname], Tomas Polasek (template)
% Description : First exercise in the Introduction to Game Development course.
%   It deals with an analysis of a selected title from the point of its genre, 
%   style, and mechanics.
%%%%%%%%%%%%%%%%%%%%%%%%%%%%%%%%%%%%%%%%%%%%%%%%%%%%%%%%%%%%%%%%%%%%%%%%%%%%%%%%

\documentclass[a4paper,10pt,english]{article}

\usepackage[left=2.50cm,right=2.50cm,top=1.50cm,bottom=2.50cm]{geometry}
\usepackage[utf8]{inputenc}
\usepackage{hyperref}
\hypersetup{colorlinks=true, urlcolor=blue}

\newcommand{\ph}[1]{\textit{[#1]}}

\title{%
Analysis of Mechanics%
}
\author{%
Antonín Masopust (xmasop05)%
}
\date{}

\begin{document}

\maketitle
\thispagestyle{empty}

{%
\large

\begin{itemize}

\item[] \textbf{Title:} The Witcher 3: Wild hunt

\item[] \textbf{Released:} 2015

\item[] \textbf{Author:} CD Project RED

\item[] \textbf{Primary Genre:} Role-Playing Game

\item[] \textbf{Secondary Genre:} Action Fighter

\item[] \textbf{Style:} Realistic Fantasy

\end{itemize}

}

\section*{\centering Analysis}

The main genre is role-playing game. It's important for this game, because it's based on a book series and let's you play as the protagonist - Geralt. It achieves the immersion with plenty of dialogues with multiple answer options. Another important aspect is that the game (as a whole and also individual quests) has several different endings, which are determined based on your decisions. Overall the story is very important for this game and is pretty well made. It's interesting, that even small quests have their own sort of unique story. Thanks to that, the player knows why they're doing what they're doing. There is a good balance between playing as a known character and having the freedom to play the game the way you like.

The second genre is action fighter, which is very fitting for this game, since the main hero is a monster slayer. The game has a complex combat system. Firstly it has a lot of keybinds dedicated for fighting (e.g. heavy and light attack, different dodges, blocks and spells). Each is useful for different opponents and occasions, making the combat diverse. There are also a lot of skills you can acquire, which are divided into three branches (swords, signs and potions) corresponding to the main play-styles. This goes well with the main genre, as it makes you feel more connected to the main character. The whole fighting experience is underlined by intense music.

The style of this game is fantasy, because it has a lot monsters, spells and unnatural objects. But is's all depicted in a realistic way with high resolution models. The physics in this game also tries to be realistic, for example waves in the water, gravity, hair, etc. I think that this approach was chosen because first, the game is open world so it let's you admire the stunning scenery and second, the realistically depicted monsters might be more scary and thus making the game more immersive. Which is beneficial for both the RPG and fighting aspect of the game. The UI is highly customizable, but in it's default setting it has two modes: combat and exploration. The combat one has more useful information like health bar, selected weapon and consumables, whereas the latter is minimalistic and less obstructive, containing just the minimap.


\subsection*{Instructions}

In this assignment, you are tasked with the analysis of a selected game-related title. The title may be a game, video game, serious game, or even serious application using game development tools. Your goal is to analyze the title from the point of its genres and style. As a part of this template, there are some placeholders and hints \ph{like this one}, which you should read and potentially replace with your own text.

\subsection*{Content}

After selecting the \ph{title}, you should first find out when it was \ph{first released} and who \ph{created it}. Be sure to consider the actual information if you choose a re-iteration or ``enhanced edition.'' 

Next, look at the game (or, even better, play it!) and determine the \ph{primary genre}. This genre should be the one supporting the core gameplay. You can use any genre taxonomy (not just the one from the lectures), but keep it unambiguous. A Game can have multiple modes of play -- e.g., Minecraft with creative and survival modes -- in which case you can choose any number of them, but be sure to emphasize your choice in the analysis.

After these steps, look at the \ph{secondary genres} and select one or more of them. Using Survival Minecraft as an example, we have a role-playing sandbox (primary) combined with the casual building and a hint of roguelike with the hardcore mode (secondary). Finally, determine the game's \ph{style} -- a combination of visual, aural, tactile, etc. For example, Minecraft can be considered a retro or cartoon-styled game.

Finally, move to the \ph{free-form text} part of the analysis in the form of short prose. Images should be used sparingly and best avoided them entirely. You should primarily focus on: 
\begin{enumerate}
    \item How are the primary and secondary genres reflected in the gameplay?
    \item How do the primary and secondary genre interact? Do the secondary genres support the primary genre? Do they enhance the game, or are they detrimental?
    \item Does the style support the gameplay? Why was it chosen?
\end{enumerate}

\subsection*{Formatting \& Submission}

Your submission must follow a similar \textbf{structure} as this template. You can either use the provided \LaTeX\ template or roughly replicate it in some other text processing software. The format of the analysis section is left up to you -- you can include sub-sections or write one long text. However, your whole document \textbf{must fit} on exactly one page of \textbf{A4}. The only accepted document format is \textbf{pdf}. Finally, you can submit the pdf by following the submission guidelines detailed on the \href{http://cphoto.fit.vutbr.cz/ludo/courses/izhv/exercises/sub/}{course's website}.

\end{document}
